שלום לכל מי שלומדים לקרוא בעברית,

\begin{framed}
\hl{בקצרה}: כדי \hl{להקל} על לימוד קריאה, \hl{צבעתי} את האותיות בכמה שירים לפי השיטה הבאה~—
\begin{compactitem}
	\item צבע כזה \symbolglyph{\a{⬢}} במילים כמו: \a{ד}\x{ג} (\symbolglyph{🐟}) או \a{צ}\x{ב} (\symbolglyph{🐢})
	\item צבע כזה \symbolglyph{\e{⬢}} במילים כמו: \e{ע}\x{ז} (\symbolglyph{🐐})
	\item צבע כזה \symbolglyph{\i{⬢}} במילים כמו: \i{פי}\x{ל} (\symbolglyph{🐘})
	\item צבע כזה \symbolglyph{\o{⬢}} במילים כמו: \o{קו}\x{ף} (\symbolglyph{🐒})
	\item צבע כזה \symbolglyph{\u{⬢}} במילים כמו: \u{סו}\x{ס} (\symbolglyph{🐎})
	\item צבע כזה \symbolglyph{\x{⬢}} כשאין תנועה. (כמו בסוף המילים האלה).
\end{compactitem}

אחרי החיות, \hl{נתאמן} קצת עם כלי־רכב:\\
\i{סי}\a{רה}~(\symbolglyph{⛵}),
\o{או}\o{טו}~/ \e{מ}\o{כו}\i{ני}\x{ת}~(\symbolglyph{🚗})
\a{מ}\o{טו}\x{ס}~(\symbolglyph{✈}),
\a{מ}\o{סו}\x{ק}~/ \e{ה}\i{לי}\o{קו}\x{פ}\e{ט}\x{ר}~(\symbolglyph{🚁}),
\a{ר}\e{כ}\e{ב}\x{ת}~(\symbolglyph{🚂}),
\a{ר}\e{כ}\e{ב}\x{ל}~(\symbolglyph{🚠}),
\a{מ}\a{ש}\i{אי}\x{ת}~(\symbolglyph{🚚}),
\o{או}\o{טו}\u{בו}\x{ס}~(\symbolglyph{🚌}),
\x{ט}\a{ר}\x{ק}\o{טו}\x{ר}~(\symbolglyph{🚜}),
\o{או}\a{פ}\a{נ}\i{יי}\x{ם}~(\symbolglyph{🚲}).

ואחרי שהתאמנו, נוכל לקרוא משפט לדוגמה:\\
\a{ה}\i{כי} \e{כי}\x{ף} \i{ל}\x{ר}\a{כ}\x{ב} \a{ע}\x{ל} \o{או}\a{פ}\a{נ}\i{יי}\x{ם}, \a{א}\a{ב}\x{ל} \a{א}\i{פי}\u{לו} \o{יו}\e{ת}\x{ר} \e{כי}\x{ף} \a{ל}\u{טו}\x{ס} \a{ל}\a{ח}\a{ל}\x{ל}.

יופי; עכשיו אפשר לדלג ישר לשירים. מקווה שתהנו \symbolglyph{☺}
\end{framed}

\hl{בחוברת הקטנה הזאת} הזאת תוכלו למצוא טקסטים\footnote{„טקסט” זאת מילה משפה שנקראת לטינית, שפעם דיברו בה בהרבה איזורים בעולם, והמשמעות שלה היא „משהו שמישהו כתב/אמר”. היא מגיעה מהמילה „טקסטוס” (בלטינית: \LR{textus}), שאומרת „אריג, משהו ארוג” (אריגה היא שיטה ליצור בד מחוטים. הרבה שטיחים הם ארוגים). זה כאילו שהמילים והמחשבות הן חוטים, ומצרפים אותן ביחד (שלובות, קלועות, שזורות, ארוגות) כדי ליצור משהו חדש, אחר, שמורכב מהן. אם אתם מכירים את המילה „טקסטורה”, תשימו לב שהיא גם מגיעה מאותו המקור. בעברית קוראים לטקטסטורה גם „מרקם”, אבל רקמה היא בכלל מלאכת־יד אחרת בחוטים. אפשר לעשות בחוטים דברים יפיפיים, שלפעמים מפתיע שהם בסך־הכל חוטים מלופפים אחד בתוך השני (ולפעמים רק חוט אחד, כמו בסריגה!).
	
	הדבר שאתם קוראים עכשיו נקרא „הערת־שוליים”, ולפעמים יש בהערות שוליים את הדברים הכי מעניינים בטקסט…} מעניינים ויפים לדעתי~— בעיקר שירים~— שאני חושב שיהיה לכם מעניין לקרוא.

בעברית הכתב לא מוסר בפשטות את כל המידע שצריך כדי לדעת איך לקרוא את הטקסט. כן מסומנות תנועות, אבל לא תמיד ולא באופן מדוייק. במונחים של בלשנות זה נקרא \hl{אבג׳ד}, וגם ערבית כותבים בכתב מסוג כזה (ולכן אפשר להשתמש בשיטה הזאת גם לערבית).

כשלומדים לקרוא טקסטים לא מנוקדים, כמו שאתם עושים עכשיו, 

השיטה היא פשוטה נורא: כל תנועה מסומנת בצבע אחר.

*** פתח גנובה: \a{ת}\u{פו}\gnuva \x{ח} (\symbolglyph{🍎})

\hl{החוברת חופשית}: זה אומר שאשמח אם תעתיקו אותה, תעשו בה שינויים ותפיצו אותה מחדש. את הטקסטים לא אני חיברתי, אבל אתם מוזמנים לעשות מה שאתם רוצים עם מה שאני עשיתי: הרעיון, הביצוע, הסימון. אפשר כמובן גם לקחת את הרעיון לטקסטים אחרים או לשפות אחרות (כשתלמדו לקרוא בערבית, אם אתם לא יודעים כבר). תוכלו למצוא את מה שעשיתי, כולל קבצים שאפשר להציג במחשב, בכתובת הזאת: \url{http://xpr.digitalwords.net/***}

גם \hl{הגופן}\footnote{„גופן” זה הצורה המסויימת של האותיות. כל אות אפשר לכתוב בסגנונות שונים; בגופן מסויים לכל האותיות סגנון דומה. לדוגמה (שימו לב לצורה השונה של האותיות אל״ף, ו״ו ות״ו בכל גופן):
	\begin{multicols}{3}
		\noindent
		{\fontspec{Guttman Hodes}אות הדסה}\\
		{\fontspec{Rutz_OE Regular Pro}אות רץ}\\
		{\fontspec{SBL Hebrew}אות סב״ל עברית}\\
		{\fontspec{Guttman Hatzvi}אות הצבי}\\
		{\fontspec{Guttman Haim}אות חיים}\\
		{\fontspec{Guttman Vilna}אות וילנא}\\
		{\fontspec{Frank Ruehl CLM}אות פרנק־ריהל}\\
		{\fontspec{David CLM}אות דוד}\\
		{\fontspec{Narkisim}אות נרקיסים}\\
		{\fontspec{Guttman Keren}אות קורן}
	\end{multicols}
	} שמשמש בחוברת הוא \hl{חופשי}. קוראים לו „אלף”, והכינו אותו כמה אנשים טובים שגם הם ישמחו אם ישתמשו בו, יעתיקו אותו וישנו אותו: מושון זר־אביב, מיכל סהר, דני מירב וניר ייני. תוכלו לקרוא עליו עוד כאן: \url{http://alef.hagilda.com/}.
