שלום,

\begin{framed}
\hl{בקצרה}: כדי \hl{להקל} על לימוד קריאה, \hl{צבעתי} את האותיות בכמה שירים לפי השיטה הבאה~—
\begin{compactitem}
	\item צבע כזה \symbolglyph{\a{⬢}} במילים כמו: \a{ד}\s{ג} (\symbolglyph{🐟}) או \a{צ}\s{ב} (\symbolglyph{🐢})
	\item צבע כזה \symbolglyph{\e{⬢}} במילים כמו: \e{ע}\s{ז} (\symbolglyph{🐐})
	\item צבע כזה \symbolglyph{\i{⬢}} במילים כמו: \i{פי}\s{ל} (\symbolglyph{🐘})
	\item צבע כזה \symbolglyph{\o{⬢}} במילים כמו: \o{קו}\s{ף} (\symbolglyph{🐒})
	\item צבע כזה \symbolglyph{\u{⬢}} במילים כמו: \u{סו}\s{ס} (\symbolglyph{🐎})
	\item צבע כזה \symbolglyph{\s{⬢}} כשאין תנועה. (כמו בסוף המילים שבדוגמאות שנתתי עכשיו).
\end{compactitem}

אחרי החיות, נתאמן קצת עם \hl{כלי־רכב}:\\
\i{סי}\a{רה}~(\symbolglyph{⛵}),
\o{או}\o{טו}~/ \e{מ}\o{כו}\i{ני}\s{ת}~(\symbolglyph{🚗})
\a{מ}\o{טו}\s{ס}~(\symbolglyph{✈}),
\a{מ}\o{סו}\s{ק}~/ \e{ה}\i{לי}\o{קו}\s{פ}\e{ט}\s{ר}~(\symbolglyph{🚁}),
\a{ר}\e{כ}\e{ב}\s{ת}~(\symbolglyph{🚂}),
\a{ר}\e{כ}\e{ב}\s{ל}~(\symbolglyph{🚠}),
\a{מ}\a{ש}\i{אי}\s{ת}~(\symbolglyph{🚚}),
\o{או}\o{טו}\u{בו}\s{ס}~(\symbolglyph{🚌}),
\s{ט}\a{ר}\s{ק}\o{טו}\s{ר}~(\symbolglyph{🚜}),
\o{או}\a{פ}\a{נ}\i{יי}\s{ם}~(\symbolglyph{🚲}).\\
\a{ה}\i{כי} \e{כי}\s{ף} \i{ל}\s{ר}\a{כ}\s{ב} \a{ע}\s{ל} \o{או}\a{פ}\a{נ}\i{יי}\s{ם}, \a{א}\a{ב}\s{ל} \a{א}\i{פי}\u{לו} \o{יו}\e{ת}\s{ר} \e{כי}\s{ף} \a{ל}\u{טו}\s{ס} \a{ל}\a{ח}\a{ל}\s{ל}.

עכשיו אפשר לדלג ישר לשירים. מקווה שתהנו \symbolglyph{☺}
\end{framed}

\hl{החוברת הקטנה הזאת} היא חוברת עם כמה טקסטים מעניינים ויפים לדעתי~— בעיקר שירים~— שאני חושב שיהיה לכם מעניין לקרוא.

בעברית הכתב לא מוסר בפשטות את כל המידע שצריך כדי לדעת איך לקרוא את הטקסט. כן מסומנות תנועות, אבל לא תמיד ולא באופן מדוייק. במונחים של בלשנות זה נקרא אבג׳ד, וגם ערבית כותבים בכתב מסוג כזה (ולכן אפשר להשתמש בשיטה הזאת גם לערבית).

השיטה היא פשוטה נורא: כל תנועה מסומנת בצבע אחר.

*** פתח גנובה: \a{ת}\u{פו}\gnuva \s{ח} (\symbolglyph{🍎})

\hl{החוברת חופשית}: זה אומר שאשמח אם תעתיקו אותה, תעשו בה שינויים ותפיצו אותה מחדש. אפשר כמובן גם לקחת את הרעיון לטקסטים אחרים או לשפות אחרות. תוכלו למצוא את מה שעשיתי, כולל קבצים שאפשר להציג במחשב, בכתובת הזאת: \url{http://xpr.digitalwords.net/***}

גם \hl{הפונט} שמשמש בחוברת הוא \hl{חופשי}, והכינו אותו כמה אנשים טובים: מושון זר־אביב, מיכל סהר, דני מירב וניר ייני. תוכלו לקרוא עליו עוד כאן: \url{http://alef.hagilda.com/}.
