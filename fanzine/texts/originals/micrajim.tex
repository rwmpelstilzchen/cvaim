יציאת מצרים של הכתב 
http://www.haaretz.co.il/1.1684926

בראשית האלף השני לפני הספירה הנוצרית המציאו כנענים, במדבר סיני, את הא"ב הראשון בעולם כולו. זהו אבי הכתב שבו אנו כותבים עד היום בעברית, באנגלית וברוב השפות המודרניות של העולם המערבי. השיטה האלפביתית הומצאה רק פעם אחת במהלך ההיסטוריה, וממנה התפתחו כל כתבי הא"ב המוכרים לנו כיום. 

מסוף האלף הרביעי ועד להמצאת הא״ב כתבו במזרח הקדום בכתבים מורכבים בני מאות סימנים. בארם נהריים~— בכתב היתדות, ובמצרים~— בכתב החרטומים, הוא כתב התמונות (ההירוגליפים) של מצרים העתיקה. מחוץ למזרח הקדום עדיין לא נמצאו עדויות חד משמעיות לכתיבה מסוג כלשהו עד האלף השני לפנה״ס.



כדי לדעת קרוא וכתוב בכתבים הטרום אלפביתיים של המזרח הקדום היה צורך להכיר ולזכור, כאמור, מאות סימנים. אולם מעבר לכך, אותם כתבים גם הובילו את הקוראים מן הסימנים אל המלים בדרכים סבוכות ומפותלות. חלק מן המלים הוצגו בכתב החרטומים על ידי ציור המתאר את משמעות המלה. למשל, המלה פר נכתבה באמצעות הציור , ואילו המלה קן נכתבה באמצעות הציור . אבל במקרה של מלים שהיה קשה למצוא ציור שיביע את מובנן במדויק (ואלו רוב המלים בשפה), היה צורך להשתמש באוסף של ציורים שלא שימשו עוד בתפקידם המקורי, אלא ציינו צליל או צירופי צלילים. כך, למשל, לו היתה המלה פרקן מלה מצרית, היא היתה יכולה להיכתב בכתב החרטומים על ידי ציור של הפר, ואחריו ציור של הקן (המתפקדים הפעם כשרשרת עיצורים בלבד~— תנועות לא היו מיוצגות). נוסף לכך היה הסופר המצרי מוסיף לפחות ציור אחד, קרוב לוודאי ציור , ציור של פפירוס כתוב חתום בחותם. זה ציור „שותק”, שאינו משתתף כלל בתבנית הפונטית של המלה. הוא משמש כסימן ממיין, המציין שהמלה הבאה לפניו היא מושג מופשט. לכן, המלה המומצאת שלנו פרקן היתה נכתבת כך:
.
אם לא די בכך, הרי שהציור יכול לתפקד בכתב גם בשימושים אחרים. גם הוא יכול להופיע כסימן ממיין חסר צליל, המציין את הקטגוריה [בקר] בשמות עצם שונים כמו פר, שור, פרה, עגל וכו׳. מכאן, שסימן יחיד כמו עשוי לשמש בשלושה תפקידים שונים במסגרת הכתב~— כתמונה של עצמו, כערך צלילי של התמונה ללא משמעותה, וכסימן ממיין חסר כל צליל.

את כתב החרטומים אפשר היה לכתוב בשני הכיוונים. המלה פרקן יכולה היתה להיכתב גם כך . הכלל היחיד הוא שהקריאה מתבצעת כנגד כיוון הסימנים, כלומר באופן שלקוראים רבים בימינו (ואולי גם בעת העתיקה) נתפש כמנוגד לכיוון הקריאה האינטואיטיבי.

הדרך החדשה לכתיבה שיצרו הכנענים במדבר סיני היתה המצאה גאונית. במקום מאות סימנים הם השתמשו בפחות משלושים סימנים בלבד, שמסמנים צלילים (ורק צלילים). באמצעות הסימנים המעטים הללו ניתן לייצג כל מלה בשפה, ובמקום להפעיל עליהם חוקי קריאה שונים ומסובכים, הא״ב מציע דרך קבועה אחת לקריאתם.

בניגוד לתפישה שרווחה עד היום במחקר, ועל פיה ממציאי הא״ב היו שייכים לאליטה חברתית־תרבותית משכילה שידעה לקרוא מצרית, אני סבורה שהם היו עובדי מכרות שלא ידעו קרוא וכתוב בשום שפה. לדעתי, העובדה שלא היה להם ידע קודם בשיטות הקריאה והכתיבה הקיימות, ושהם פעלו מחוץ לעולם הממסדי של תרבויות המזרח הקדום, היא שסייעה להם ליצור את השיטה החדשה והמקורית, שהיתה פשוטה ונגישה הרבה יותר מן השיטות שקדמו לה.

יוצרי הכתב החדש היו כנענים~— היינו אנשים ששפתם היתה כנענית, שפה שדיאלקטים שונים שלה דוברו בכל אזור הלבנט בתקופה ההיא. הם עבדו בשירות המלך המצרי במכרות הטורקיז והנחושת בדרום הר סיני, באזור סראביט אל חאדם. שם, בסביבות 1840 לפני הספירה הנוצרית, כלומר כמעט לפני 4000 שנה, הם המציאו את הא״ב.

פרעוני מצרים שלחו משלחות גדולות לראש ההר בסראביט אל חאדם. במשלחות היו כמובן כורים, אבל גם סופרים מצריים, פקידי אוצר ורופאים, חיילים, אומנים למלאכות רבות כמו בנאים וסתתים, וכן מתורגמנים, מובילי חמורים ואפילו מכשפי עקרבים (מכשפים שתפקידם היה כנראה למנוע הכשת עקרבים). נוסף לעבודת הכרייה עסקו המצרים גם בבניית מקדש גדול על ראש ההר ל„בעלת הטורקיז”~— הלוא היא האלה חתחור, שקיוו תמיד לברכתה~— ולכן נזקקו למערך עובדים גדול ומגוון כל כך.

במקדש הגדול על ראש ההר נמצאו מאות כתובות בכתב התמונות הקטנות, כתב החרטומים, המספרות על המשלחות ועל הצלחתן בשל זכייתן בברכת האלים. כתובות מצריות הירוגליפיות לא מעטות התגלו גם בסביבת המכרות, במרחק לא גדול מן המקדש. ככל שאנו יכולים להסיק מן הכתובות הרבות שהותירו המצרים במקדש ובסביבתו, העובדים במקדש ובמכרות לא היו עבדים. מעבר לכך, מן הכתובות המצריות אנו למדים שבמשלחות היו כנענים רבים שעבדו עם המצרים בתפקידים שונים. הם לא שימשו רק ככורים, אלא היו ביניהם גם מובילי שיירות, מנהלי עבודה, בנאים וחיילים. מוכר לנו אפילו נסיך כנעני בשם חבדד, שמופיע עם חמורו ונושאי כליו בציור שהוא חלק מאסטלה מצרית שנחשפה במקדש.

אולם, גם כתובות אחרות התגלו על ראש ההר לצד מאות הכתובות בכתב החרטומים. הילדה פיטרי, אשתו של הארכיאולוג הנודע פלינדרס פיטרי, שחפר באתר ב־1905, דרכה באקראי ליד אחד המכרות על כמה שברי אבנים, שנשאו סימנים מוזרים ו„מכוערים” במיוחד. הסימנים הללו נראו כמו חיקויים לא מוצלחים של הירוגליפים מצריים, אך פיטרי חד האבחנה הציע כבר אז שמדובר בכתב אלפביתי, אף שלא היה מסוגל לקוראו. הכתב פוענח ב־1916 על ידי האגיפטולוג האנגלי הנודע אלן גרדינר, שזיהה את שפת הכתובות ככנענית. מאז נמצאו ותועדו באזור המקדש, המכרות והדרך אליהם כשלושים כתובות באותו כתב אלפביתי.

רוב הכתובות נמצאו מסביב למכרות הטורקיז ואפילו ממש בתוכם, ורק כמה כתובות נמצאו על ארבעה חפצי מנחה קטנים במקדש עצמו. בשל מיקומן של הכתובות אפשר להניח שנכתבו על ידי העובדים במכרות ולא על ידי סופרים או בעלי תפקידים בכירים במקדש, כפי שמקובל לחשוב בקרב החוקרים.

קיים דמיון רב בין כמה סימנים בכתובות הללו לבין הירוגליפים מצריים מסוימים שמופיעים בכתובות במקדש לאלת הטורקיז, המתוארכות לימי המלך אמנאמחת השלישי (בסביבות 1840לפנה״ס). הקשרים הברורים בין הסימנים החדשים לבין אותם הירוגליפים מוכיחים לדעתי בסבירות גבוהה מאוד שהמודלים לאותיות החדשות של הכנענים נלקחו מתוך רפרטואר הסימנים האופייני לסיני בתקופתו של מלך זה. בעקבות זאת, ובשונה מן התפישה הרווחת במחקר (לבד מזו של גרדינר, מפענח הכתב), אפשר להסיק שהא״ב הזה הומצא באותו מקום ובאותו זמן. כמו כן, מעט הכתובות האלפביתיות הנוספות שנמצאו במקומות אחרים במצרים ובכנען, מאוחרות כולן לתקופה זו; דבר שגם הופך, כאמור, את הא״ב מסיני לא״ב הראשון בעולם.

למרות הדמיון שבין האותיות של הכתב החדש להירוגליפים, ברור לדעתי שממציאי הא״ב לא ידעו לקרוא מצרית (או הכירו מעט מאוד סימנים). אחת הדרכים להסיק זאת היא שהם השתמשו בכמה סימנים מצריים שונים, שדמו זה לזה בצורתם, כמודלים לאותה אות. למשל, אפשר לראות שלאות נ׳, שימשו כדגם שני הירוגליפים שונים של נחשים – אחד של קוברה והשני של שפיפון . במצרית אין שום אפשרות להחליף בין שני ציורי הנחשים האלה, מכיוון שהם בעלי משמעות שונה ונקראים באופן שונה לחלוטין. אך לגבי הכנענים נחש היה פשוט נחש.

נוסף לכך, הכתובות הכנעניות כתובות בכיוון הפוך לכיוון הקריאה הנכון במצרית. כפי שראינו למעלה, סימנים מצריים נקראים תמיד כנגד כיוון הציורים. אך הכותבים הכנענים בחרו בכיוון ה„אינטואיטיבי”, ההפוך, כאמור, לכיוון הקריאה התקני במצרית. כמו כן, בניגוד למסורת הכתיבה של הסופרים המצרים, רוב הכתובות הכנעניות מאופיינות בחוסר הקפדה מוחלט על סדר, גודל האותיות, סידור השורות וכו׳ (ראו פסלו של נעם).



מה היו אם כן התנאים שהובילו להמצאת הכתב הזה דווקא בסיני על ידי עובדי המכרות הכנענים? הכנענים ה„אנאלפביתים” ראו מסביבם כתובות רבות בכתב החרטומים. המצרים שהקיפו אותם היו אחוזים בדיבוק של כתיבה. הכנענים ידעו כי בעזרת התמונות הקטנות הללו מתקשרים המצרים עם האלים ומספרים בשבח עצמם. אולי הם אפילו ייחסו את ההצלחה המצרית לתקשורת הטובה שלהם עם האלים באמצעות כתב התמונות. הם ידעו שאפשר לכתוב לאלים ולבקש את ברכתם. בעיקר נמשכו אל האפשרות להותיר שם על אבן. כך לא יימחק שמם; האלים יזכרו אותם והם יבורכו לעולם.

העבודה במכרות החשוכים היתה מפרכת ומסוכנת. האמונה בנוכחות האלים ובשליטתם בגורל האישי היתה מוחשית וצורבת על ראש ההר הצחיח בסיני. הרצון ליצור עמם קשר ולבקש את ברכתם היה בוודאי צורך קיומי. הכנענים ביקשו לכתוב לאליהם שלהם – לבעלת (שמה הכנעני של אלת הטורקיז, חתחור), ולאבי האלים הכנעני ששמו בתקופה זו היה אל.

הכנענים אימצו לעצמם כשני תריסר ציורים בלבד מהכתב המצרי בן מאות התמונות. הם בחרו מכתב החרטומים תמונות שדיברו אל לבם – כמו למשל שור (אלף בכנענית, במקרא ידועה צורת הריבוי אלפים), עין , בית , מים , ראש או יד .

כיוון שלא הכירו את חוקי הכתב המצרי המסובך, עשו בהירוגליפים שימוש שונה ומקורי לחלוטין, והמציאו בהשראתם כתב חדש כדי לכתוב את שפתם הם – הכנענית העתיקה. הנה כמה דוגמאות:

את ההירוגליף המצרי הם זיהו כתמונה של המלה „ראש” בכנענית, והחליטו להשתמש בו כדי לבטא רק את הצליל הראשון של המילה („ראש” = ר).

מעתה, בכתב הכנעני, הצליל של הציור יהיה ר ולא ראש. הציור נעשה „סימן חופשי” שאינו כבול עוד למשמעות התמונה, ואפשר יהיה להשתמש בו בכל פעם שמבקשים לייצג את הצליל ר.

במצרית, המלה ראש נשמעה אחרת לגמרי, כנראה נהגתה כ„טפ” או באופן דומה. אבל הכנענים לא ידעו זאת, או שזה לא היה חשוב להם.

הכנענים לא למדו, כאמור, כיצד לכתוב במצרית ולצייר במדויק את ההירוגליפים (עניין שדרש מיומנות). זו כנראה אחת הסיבות שבמעבר מן הכתב המצרי לא״ב הכנעני השתנתה בחלק מן המקרים גם צורתו של הסימן. ניתן להבחין בכך כשמתבוננים, למשל, בסימן הכנעני הזה, בהשוואה להירוגליף ששימש לו השראה. בגרסאות מאוחרות יותר של ציור הראש הכנעני, ניתן להבחין שהוא גם עבר התאמה ל„אופנה” הכנענית, כך שהתסרוקת המתנוססת על הראש מזכירה מאוד את האופן שבו נהגו הכנענים לספר את שיערם (תסרוקת הפטרייה) .

הירוגליף מצרי אחר, שצורתו ריבוע זיהו הכנענים כחתך פשוט של בית. שם האות הכנענית הוא בית, אבל גם במקרה הזה השתמשו הממציאים בציור כדי לייצג רק את הצליל הראשון של המילה בית. מעתה הפך גם הוא ל„סימן חופשי” שאינו כבול למשמעות התמונה. במצרית הצליל של אותו סימן היה כנראה פּוֹי ומשמעותו היתה בכלל שרפרף, אך גם הפעם לא היו הכנענים מודעים לכך, או שבחרו להתעלם מן העובדות הללו.

את ההירוגליף המצרי~— שייצג במצרית את הפועל שמשמעותו היתה לעשות, והצליל שלו היה אִרִי~— זיהו הכנענים כתמונה של המלה „עין” בשפתם. כמו במקרים האחרים הם בחרו להשתמש רק בצליל הראשון של המלה „עין” = ע, ובכתב הכנעני מבטא הציור את הצליל ע בלבד.

את ההירוגליף המצרי שמופיע בכתובות מצריות רבות מסיני באותה תקופה, הם פירשו כנראה כתמונה של איש הקורא „הוי!” (אולי בדומה לקריאתו של מנהל העבודה במכרות), ומכאן שמה של האות „הא”.

הממציאים הכנענים גם הרחיבו בכמה מקרים את השיטה החדשה מעבר לסימנים שראו סביבם. הם בחרו תמונות משלהם, שאינן חלק מרפרטואר הכתב המצרי. כך, למשל, ציור האות הכנענית „כף” אינו מוכר בכתב החרטומים.

רוב שמות האותיות בעברית של ימינו קשורים עדיין למקור העתיק של הא״ב, ודוברי העברית יכולים להבין את משמעותם. הם שומרים על שמות האותיות הכנעניות העתיקות, ובעצם מתארים את צורתן.הבית הלא היא ה„בית” העתיק, הדלת הרי היא הדֶלֶת, הוו היא וו (סיכת בגד כנעני), וכך גם היוד (יד), הכף (כף יד), המם (מים), העין (עין), הריש (ראש) וכן הלאה.

ברור ששמות אלה אפשרו במשך מאות שנים לנושאי הכתב החדש~— כנענים שהיו אנשי שיירות, חיילים, כורים, חוצבים וסוחרים ולא כותבים מקצועיים~— לזכור את צורת האותיות ולשחזר אותן בכל פעם מחדש. עד המאה ה־12 לפני הספירה אנו מכירים את הכתב הזה רק מכתובות קצרות מאוד של שמות וברכות לאלים, מה שמעיד על כך שהכתב לא שימש לצורכי מינהל, אלא עדיין שמר על תפקידו כמנציח שמות ומקשר בין האדם לבין האל.

בהיסטוריה של הרעיונות, מהפכת התקשורת הזאת מעניינת מכמה בחינות. ראשית, „מנוע ההמצאה” ששינתה את פני ההיסטוריה היה – אם שחזורי נכון – דתי ורגשי, ולא נבע מצורך של מדינה או אדמיניסטרציה לבנות מערכת שתאפשר גביית מסים או שליטה טובה יותר במשאבי המדינה והתוצר, כמו במקרים של כתב היתדות וכתב החרטומים במסופוטמיה ובמצרים.

שנית, הא״ב הוא דוגמה להמצאה מבריקה של קבוצה חלשה יחסית בחברה, שהיתה רחוקה מן המרכזים של התקופה. הקבוצה הזאת הצליחה לשמר את החידוש במשך מאות שנים הודות לנגישות ולפשטות של ה„חידוש הטכנולוגי”.

אולם, מהפכות טכנולוגיות אינן מובילות בהכרח לשינוי תרבותי מהיר. יתרונותיה של ההמצאה נתבררו רק כשחלק מנושאיה נהפכו לשחקנים הראשיים על בימת ההיסטוריה. בסוף האלף השני לפנה״ס ירדו מגדולתן מעצמות המזרח הקדום, ואתן גם הערים הגדולות בכנען על סופריהן ששימרו את מסורות הכתיבה בכתב היתדות ובכתב החרטומים. לחלל שנוצר נכנסו בהדרגה קבוצות השוליים הכנעניות: איכרים, נוודים ואחרים, שעתידים להיקרא לימים עברים, עמונים, מואבים, פניקים וארמים. הם עתידים להקים את הממלכות הכנעניות החדשות, שנהפכו לישויות המדיניות המובילות באזור. הכתב הרשמי של כל הממלכות הללו היה באופן טבעי הא״ב הכנעני הקדום, ומהן הוא עבר ליוונים ואחר כך לעולם המערבי כולו.

היום כותבים עברית בכתב הארמי, שגם הוא תולדה של הא״ב העתיק מסיני, אף על פי שצורת אותיותיו השתנתה מאוד. אך למרות התהפוכות שעבר הכתב שלנו מאז צאתו ממצרים, הרי שכמעט בכל אות עברית שאנו כותבים היום, נם לו הירוגליף מצרי עתיק.

לזכרו של יוסף נוה. תודה לשי צור שסייעה בכתיבת הגירסה העברית של המאמר. תודה לדן אלהרר ולהללי הראל על עזרתם בהכנת התמונות 
