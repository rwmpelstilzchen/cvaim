\textsection{מעמקי האוקיינוס}

הימים והאוקיינוסים מכסים שני שלישים משטח פניו של כדור־הארץ והם בית־הגידול הטבעי הגדול ביותר בו. האוקיינוסים עצומים ומגוונים, ונופה של קרקעית הים מסעיר יותר מנופן של היבּשות. יש בה הרים מתנשׂאים ותהומות עמוקים ואפלים. במקומות מסוימים ההרים התת־מימיים גבוהים כל־כך עד שהם יוצרים איים מעל פני המים. אולם, רוב סודותיהם של האוקיינוסים עדיין אינם ידועים לנו. האוקיינוסים נמנים עם אחרוני המקומות על פני כדור־הארץ שבּני־האדם עדיין לא חקרו אותם חקירה ממשית, וייתכן מאוד שיש בהם חיות רבּות שעדיין לא נתגלו.

גם על־פי הידיעות שיש בידינו היום חיים באוקיינוסים מיגוון עצום של יצורים בממדים שונים~— למן הפּלאנקטון המיקרוסקופּי ועד הלווייתן הענק. רוב החיות מתגוררות בחלקים העליונים, המוארים באור שמש, של מי הים~— באזור שהמזון הצמחי מצוי בו בשפע. באזור זה „רועות” להקות גדולות של דגים קטנים, המשמשים טרף לדגים גדולים יותר. ואולם, בעומק קילומטרים אחדים מתחת לאזור הזה, המים שחורים כזפת וקרים עד להקפּיא. במעמקים האלה אין קיום לשום צמח, ואף־על־פי־כן חיים שם המוני דגים מוזרים. רבּים מהם מספּקים לעצמם תאורה באמצעות מערכים מסתוריים של קוצים וסנפּירים זוהרים.



\textsection{מה הוא מה בעולם החי}

מספּרם של מיני החיות בעולם רב כל־כך~— יותר מ־10 מיליון~— עד שחוקרי הטבע נאלצו לסווג אותם לפי קבוצות. בלא החלוקה הזאת הם היו מתקשים לחקור את המינים ולקבּוע את הקשרים ביניהם. \rewritingdelete{לכל המינים הנמנים עם קבוצה כלשהי יש לפחות תכונה אחת משותפת~— עמוד־שדרה, למשל, או עמוד־שדרה ואותו מספּר רגליים.} \rewritingdelete{חלק מן} הקבוצות מחולקות לתת־קבוצות. בעמודים האלה מופיעות \rewriting{כמה} קבוצות בעלי־החיים \rewritingdelete{ה}עיקריות.


\textsubsection{חסרי חוליות}

אחת ההבחנות הראשונות בחלוקת החיות לקבוצות מבוססת על קיומו או אי־קיומו של עמוד־שדרה. חסרי חוליות הם בעלי־חיים שאין להם עמוד־שדרה. לחלק מחסרי החוליות~— למשל, למדוזות~— יש גוף רך. לאחרים~— למשל, לחרקים~— יש שלד חיצוני קשיח. חרקים, עכּבישניים, סרטנים ורכּיכות הם ארבּע קבוצות של חסרי חוליות.


\textsubsection{חרקים}

החרקים הם קבוצת החיות הגדולה ביותר על פני כדור־הארץ. אפשר למצוא חרקים בכל מקום. לכל חרק יש שש רגליים וגוף הנחלק לשלושה חלקים. לרוב החרקים יש זוג כנפיים אחד או שני זוגות כנפיים. לקבוצת החרקים שייכים, בין השאר, הפּרפּרים, החיפּושיות, הדבורים והנמלים.


\textsubsection{עכּבישניים}

לעכּבישניים יש שמונה רגליים (שתי רגליים יותר ממה שיש לחרקים) ושלד חיצוני נוקשה. בגופם יש על־פי רוב שני חלקים. רוב העכּבישניים הם ציידים החיים על פני היבּשה. לקבוצה זו שייכים, בין השאר, עכּבישים, עקרבּים, קוצרים ואקריות.


\textsubsection{סרטנים}

רוב הסרטנים חיים במים, ובעיקר במי ים. עם אלה נמנים לובּסטרים, סרטנים קצרי־בטן וסרטנים ארוכּי־בטן. סרטנים אחרים~— למשל, טחביות~— מתגוררים במקומות לחים על פני היבּשה. לכולם גוף המכוסה בקליפּה קשה. קליפּה זו בנויה לעתים מלוחות מאוחים.


\textsubsection{רכּיכות}

\rewritingdelete{מלפפוני־ים, }חלזונות, צדפות ודיונונים הם כולם רכּיכות. הרכּיכות ניכּרות בגופן הרך, המכוסה לעתים בקליפּת מגן קשה.\rewritingdelete{שלא כמו הסרטנים, הרכּיכות חסרות רגליים.} הן חיות בעיקר במים והן קבוצת החיות השניה בגודלה.


\textsubsection{חולייתנים}

החולייתנים (בעלי החוליות) הם חיות בעלות עמוד־שדרה ושלד פנימי. כיום אנו מכּירים בערך 46,000 מיני חולייתנים. חוקרי הטבע מחלקים אותם לחמש קבוצות עיקריות: דגים, דו־חיים, זוחלים, עופות ויונקים. יותר ממחצית מיני החולייתנים הידועים לנו הם דגים.


\textsubsection{דגים}

הדגים הם יצורים בעלי דם קר, כלומר: טמפּרטורת הגוף שלהם זהה לטמפּרטורה של סביבתם. הדגים מגוונים מאוד בצורתם, אבל כמעט לכולם יש סנפּירים וכמעט כולם חיים במים ונושמים באמצעות זימים. חלקם חיים במים מתוקים, וחלקם במי ים.


\textsubsection{דו־חיים}

הדו־חיים הם יצורים בעלי דם קר ועור חלק ולח שאין בו נוצות או קשׂקשׂים. הם יכולים לחיות על פני היבּשה \rewriting{ובמים} כאחד, אבל צריכים לשוב אל המים בעונת הרבייה. עם קבוצה זו נמנים, בין השאר, צפרדעים, קרפּדות, סלמנדרות וטריטונים.


\textsubsection{זוחלים}

הזוחלים הם יצורים בעלי דם קר ועור קשׂקשׂי. הם מתגוררים על פני היבּשה ובתוך המים כאחד. נקבות הזוחלים מטילות ביצים. הזוחלים הקטנים הבּוקעים מתוכן אינם צריכים לעבור שום גלגול~— להבדיל מהדו־חיים. לקבוצה זו שייכים, בין השאר, נחשים, לטאות, צבּים ותנינים. זוחלים רבּים הם טורפים האוכלים חיות אחרות.


\textsubsection{עופות}

העופות הם יצורים בעלי דם חם, כלומר: גופם שומר על טמפּרטורה קבועה בלי שום קשר לטמפּרטורת הסביבה. הם בעלי־החיים היחידים שגופם עטוי נוצות (אבל רגליהם מכוסות בקשׂקשׂים). לכל העופות יש כנפיים ורובּם יכולים לעופף.


\textsubsection{יונקים}

עם קבוצת היונקים נמנים מינים שונים במיגוון עצום של צורות וממדים~— למן הדולפין ועד העטלף, ולמן הפּיל ועד הכּלב. היונקים הם יצורים בעלי דם חם והיחידים שיש להם שׂיער או פרווה. צאצאיהם יונקים את חלב אמם (ומכּאן שמם). היונקים מצטיינים בחושים ובמוחות מפותחים. \rewriting{שלוש קבוצות של יונקים שנפרט עליהן כאן} הן מכרסמים, חיות־כיס ופרימאטים.


\textsubsection{מכרסמים}

כמעט מחצית היונקים המוכּרים לנו הם מכרסמים. מקור שמם בדרך האכילה שלהם. לכל המכרסמים שיניים קדמיות חדות. לקבוצה זו שייכים, בין השאר, עכבּרים, בונים, טמיאסים וסנאים. חלק מן המכרסמים אוכלים צמחים, וחלקם אוכלים נבלות.


\textsubsection{חיות־כיס}

קנגורים, וומבּטים וקואלות הם כולם חיות־כיס (כיסאים). לחיות־הכּיס הנקבות יש „כיסים” מיוחדים על הבּטן. הוולדות של חיות־הכּיס עדיין לא השלימו את התפּתחותם בעת שהם יוצאים לאוויר העולם, ולכן הם ממשיכים לגדול בכיסה החם של אמם.


\textsubsection{פרימאטים}

למורים, גלאגו, קופים וקופי־אדם \rewriting{(כולל בני־אדם)}~— כל אלה הם פרימאטים. רובּם שוכני עצים המתגוררים ביערות הגשם הטרופּיים. תכונותיהם המשותפות הן המראה, הידיים הלופתות, הראייה החדה והתבונה, העולה על זו של רוב היונקים האחרים.
