האופציה של העתקת תוכנית ותיקונים היא האופציה הפשוטה והישירה ביותר להעברת טכנולוגיה, אך לא תמיד היא אפשרית. תוכניות יש שהן נשמרות בסוד, ולפעמים אין הן מובנות למי שאינו בקיא מלכתחילה בטכנולוגיה. יש שנפוצות שמועות על המצאה חדשה שהומצאה באיזה מקום רחוק, אבל פרטיה אינם מועברים. אולי ידוע רק הרעיון הבסיסי: מישהו הצליח איכשהו להשיג תוצאה סופית מסוימת. הידיעה הזאת יכולה בכל זאת לעורר אחרים, על ידי התפשטות רעיון, להמציא נתיבים משלהם אל התוצאה הזאת.

דוגמה בולטת מתולדות הכתב היא המצאת הכתב ההברתי לכתיבת השפה הצ׳רוקית בידי אינדיאני צ׳רוקי ששמו סקוויה בארקנזס בסביבות 1820. סקוויה ראה שהלבנים רושמים סימנים על נייר, וכי השימוש בסימנים האלה מסייע להם לרשום על הנייר ולקרוא מן הנייר נאומים ארוכים. ואולם פרטי דרך הפעולה של הסימנים האלה נשארו חידה בעיניו, כי (כמו רוב הצ׳רוקים עד 1820) סקוויה היה אנאלפבית ולא ידע לא לדבר ולא לקרוא אנגלית. מאחר שהיה נפח, המציא תחילה שיטת חישוב שתסייע לו לעקוב אחר חובות לקוחותיו. לכל לקוח צייר תמונה, ולידה מעגלים וקווים בגדלים שונים לציין את סכום הכסף שהוא חב לו.

בסביבות 1810 החליט סקוויה להמשיך הלאה ולהמציא שיטה לכתיבת השפה הצ׳רוקית. תחילה צייר שוב תמונות, אבל עד מהרה זנח את הדרך הזאת כי היתה מסובכת מדי ותובענית מדי מבחינה אמנותית. הצעד הבא היה המצאת סימנים נפרדים לכל מילה, ושוב לא נחה דעתו, כי טבע אלפי סימנים ולא היה בהם די.

לבסוף הבין סקוויה שהמילים מורכבות ממספר לא רב של יחידות צליל שחוזרות בהרבה מילים שונות~— מה שקרוי בפינו הברות. תחילה המציא 200 סימנים הברתיים, ולאט לאט צמצם אותם ל־85, רובם צירופים של עיצור אחד ותנועה אחת.

אחד ממקורות הסימנים היה ספר איות באנגלית שקיבל סקוויה ממורה של בית ספר והתאמן בהעתקת אותיותיו. כשני תריסרי סימנים בכתב ההברתי הצ׳רוקי נלקחו ישירות מאותן אותיות, אף שכמובן ניתנו להן משמעויות שונות לחלוטין, שכן סקוויה לא הכיר את משמעויותיהן האנגליות. למשל, הוא בחר את האותיות \LR{D}, \LR{R}, \LR{b} ו־\LR{h} לייצג את ההברות הצ׳ירוקיות a, e, si ו־ni, לפי סדר זה, ואילו סימן המספר 4 הושאל להברה se. סימנים אחרים היו האותיות האנגליות בשינויים קלים, כגון הסימנים \cherokee{Ᏻ}, \cherokee{Ꮜ} ו־\cherokee{Ꮎ} המציינים את ההברות yu, sa ו־na, לפי סדר זה. סימנים אחרים המציא סקוויה בעצמו, כגון \cherokee{Ꮀ}, \cherokee{Ꮅ} ו־\cherokee{Ꮔ} ל־ho, li, ו־nu. הכתב ההברתי של סקוויה מעורר הערצה רבה בקרב בלשנים מקצועיים בזכות התאמתו הטובה לצלילים הצ׳רוקיים וקלות לימודו. בתוך זמן קצר הגיעו הצ׳רוקים כמעט ל־100 אחוזי אוריינות בכתב ההברתי, קנו מכונות דפוס, יצקו אותיות על פי סימניו של סקוויה, והחלו בהדפסת ספרים ועיתונים.

הכתב הצ׳רוקי הוא אחת הדוגמאות המתועדות ביותר לכתב שצמח בהתפשטות רעיון. ידוע לנו שסקוויה קיבל נייר וכלי כתיבה אחרים, וכן קיבל את הרעיון של שיטת כתב, את הרעיון של שימוש בסימנים נפרדים, ואת הצורות של כמה עשרות סימנים. ואולם מאחר שלא ידע לקרוא ולכתוב אנגלית, לא קנה לו מהכתבים הקיימים שראה סביבו שום ידיעה על פרטיהם, ואף לא על עקרונותיהם. מאחר שהיה מוקף באלפבתים שלא היה בכוחו להבינם, שב והמציא באופן עצמאי כתב הברתי, ולא ידע שהמינואים של כרתים הקדימוהו והמציאו כתב הברתי אחר לפני 3,500 שנה.

הדוגמה של סקוויה יכולה לשמש דגם לדרך שבה נוצרו שיטות כתב רבות עקב התפשטותו של רעיון גם בימי קדם. האלפבית ההנגולי שהמציא מלך קוריאה סג׳ונג ב־1446 לסה״נ בשביל השפה הקוריאנית ניכר שהושפע מדגם הגוש של האותיות הסיניות ומן העקרון האלפביתי של הכתיבה הבודהיסטית במונגוליה או בטיבט. ואולם המלך סג׳ונג המציא את צורות האותיות ההנגוליות וכמה מן התכונות הייחודיות של האלפבית שלו, ובכללן הקבצת האותיות לפי הברות בגושים ריבועיים, השימוש בצורות אותיות קרובות לייצוג צלילי תנועות או עיצורים קרובים, וצורת האותיות העיצוריות שמתארת את מצב השפתיים או הלשון בזמן ביטוי העיצור. האלפבית האוגמי שהשתמשו בו באירלנד ובחלקים מבריטניה הקלטית מסביבות המאה ה־4 לסה״נ אימץ גם הוא את העקרון האלפביתי (במקרה זה, מהאלפביתים האירופיים שהיו קיימים אז), אך גם כאן, הומצאו צורות ייחודיות של אותיות, כנראה על יסוד שיטת חמש־אצבעות של סימני ידיים.

אפשר לקבוע בבטחה ששיטות הכתב האלפביתי אוגם והנגול נוצרו עקב התפשטות רעיון ולא הומצאו באופן עצמאי בבידוד, כי ידוע לנו ששתי החברות באו במגע קרוב עם חברות בעלות כתב ומפני שברור מאילו כתבים זרים הושפעו. לעומת זה, אנו יכולים לקבוע בבטחה שכתב היתדות השומרי והכתב הקדום של אמריקה התיכונה היו פרי המצאה עצמאית, כי כשהופיעו לראשונה לא היתה בשני חצאי כדור הארץ שום שיטת כתב אחרת שהיתה יכולה להשפיע עליהם. עדיין שנוי במחלוקת מקור הכתב באי הפסחא, בסין ובמצרים.
