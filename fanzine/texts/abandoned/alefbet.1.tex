השתלשלות כתבים מהכתב הפרוטו־סינאי

נראה שמקורם של רוב סוגי האלפבית (כתב שבו הסימנים הכתובים מייצגים צלילים) ברחבי העולם בכתב אחד או במספר מצומצם של כתבים. ההשערה הרווחת כיום במחקר ההיסטוריה של הכתב היא שרוב האלפביתים בעולם השתלשלו מהכתב הפרוטו־סינאי, שהוא האלפבית הקדום ביותר שנמצא עד כה. ביניהם: האלפבית הלטיני (\L{ABC}), שממנו נגזרו אלפביתים רבים בשימוש כיום, העברי (אבג), הערבי (ابت), הקירילי (абв), הרוני (ᚠᚢᚦ), הגיאורגי (აბგ), האתיופי (ሀለሐ) והדוונגרי (कखग).

קיימים אלפביתים שלא השתלשלו מהאלפבית הפרוטו־סינאי, כמו הכתב הקוריאני „האנגול” (ㄱㄴㄷ) והכתב הפונטי הסיני ג'ו-יין הומצאו באחת על ידי שליט.

הכתב הפרוטו-סינאי נגזר ברובו מכתב הציורים המצרי, שעבר הסדרה פונטית לכדי אותיות, בתהליך אקרופוניה. הכתב המרואיטי, שהוא ככל הנראה אלפבית עצמאי, נגזר גם הוא מההירוגליפים המצריים, אך בצורה רופפת יותר, ועל כן יש הסוברים שגם הוא מקורב ל"שושלת הפרוטו-סינאית".
